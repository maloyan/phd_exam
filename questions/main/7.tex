\section{
    Собственные и несобственные интегралы, зависящие от параметра. \\
    Равномерная сходимость по параметрам и ее признаки. \\
    Непрерывность, интегрирование и дифференцирование интегралов по параметру.
}

1. \textbf{Собственный интеграл}

Определенный интеграл называется собственным, если область интегрирования и функция интегрирования являются ограниченными

2. \textbf{Несобственный интеграл}

Определённый интеграл называется \textbf{несобственным}, 
если выполняется по крайней мере одно из следующих условий:
\begin{itemize}
    \item Область интегрирования является бесконечной ${\displaystyle [a,+\infty )}$.
    \item Функция ${\displaystyle f(x)}$ является неограниченной 
    в окрестности некоторых точек области интегрирования.
\end{itemize}

\textbf{Несобственный интеграл I рода} ${\displaystyle \exists \lim _{A\to +\infty }\int \limits _{a}^{A}f(x)dx=I\in \mathbb {R} }$

\textbf{Несобственный интеграл II рода} ${\displaystyle \forall \delta >0\Rightarrow \exists \int \limits _{a+\delta }^{b}f(x)dx={\mathcal {I}}(\delta )}$ или 
${\displaystyle \exists \int \limits _{a}^{b-\delta }f(x)dx={\mathcal {I}}(\delta )}$.
При этом
${\displaystyle \exists \lim _{\delta \to 0+0}{\mathcal {I}}(\delta )=I\in \mathbb {R} }$

3. \textbf{Интеграл, зависящий от параметра}

Пусть в двумерном евклидовом пространстве задана область ${\displaystyle {\overline {G}}=\left\{\left(x,y\right)|a\leq x\leq b,c\leq y\leq d\right\}}$
на которой определена функция ${\displaystyle f(x,y)}$ двух переменных.

Пусть далее, ${\displaystyle \forall y\in \left[c;d\right]\,\exists I\left(y\right)=\int \limits _{a}^{b}f\left(x,y\right)\,dx}$.
Функция ${\displaystyle I(y)}$ и называется интегралом, зависящим от параметра.

4. Свойства

\begin{itemize}
    \item \textbf{Непрерывность} 

    Пусть функция ${\displaystyle f(x,y)}$ непрерывна в области 
    ${\displaystyle {\overline {G}}}$ как функция двух переменных. 
    Тогда функция ${\displaystyle I\left(y\right)=\int \limits _{a}^{b}f\left(x,y\right)\,dx}$ 
    непрерывна на отрезке ${\displaystyle [c;d]}$.

    \item \textbf{Дифференцирование под знаком интеграла}
    Пусть теперь на области ${\displaystyle {\overline {G}}}$ 
    непрерывна не только функция ${\displaystyle f(x,y)}$, 
    но и её частная производная ${\displaystyle {\frac {\partial f}{\partial y}}\left(x,y\right)}$.
    Тогда ${\displaystyle {\frac {d}{dy}}I(y)=\int \limits _{a}^{b}{\frac {\partial f}{\partial y}}\left(x,y\right)\,dx}$

    \item \textbf{Интегрирование под знаком интеграла}

    Если функция ${\displaystyle f(x,y)}$ непрерывна в области ${\displaystyle {\overline {G}}}$, то
    ${\displaystyle \int \limits _{c}^{d}I(y)\,dy=\int \limits _{a}^{b}\left(\int \limits _{c}^{d}f(x,y)\,dy\right)\,dx}$
\end{itemize}