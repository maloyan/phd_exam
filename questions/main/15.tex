\section{
    Линейная зависимость и независимость векторов. Ранг матрицы. \\
    Системы линейных алгебраических уравнений, теорема Кронекера-Капелли. \\ 
    Общее решение системы линейных алгебраических уравнений. 
}

1. Линейная зависимость и независимость векторов

Конечное множество ${\displaystyle M'=\{v_{1},v_{2},...,v_{n}\}}$ называется линейно
 независимым, если единственная линейная комбинация, равная нулю, тривиальна, 
 то есть состоит из факторов, равных нулю:
${\displaystyle a_{1}v_{1}+a_{2}v_{2}+\ldots +a_{n}v_{n}=0\quad \Rightarrow \quad a_{1}=a_{2}=\ldots =a_{n}=0.}$

Если существует такая линейная комбинация с минимум одним
${\displaystyle a_{i}\neq 0}a_{i}\neq 0, {\displaystyle M'}$ называется линейно зависимым.

2. \textbf{Ранг}

Пусть ${\displaystyle A_{m\times n}}$ — прямоугольная матрица. Рангом матрицы ${\displaystyle A}$ является:
\begin{itemize}
    \item ноль, если ${\displaystyle A}$ — нулевая матрица;
    \item число ${\displaystyle r\in \mathbb {N} :\;\exists M_{r}\neq 0,\;\forall M_{r+1}=0}$, 
    где ${\displaystyle M_{r}}$ — минор матрицы ${\displaystyle A}$ порядка ${\displaystyle r}$, 
    а ${\displaystyle M_{r+1}}$ — окаймляющий к нему минор порядка ${\displaystyle (r+1)}$, 
    если они существуют.
\end{itemize}

3. Теорема Кронекера-Капелли

Система линейных алгебраических уравнений совместна тогда и только тогда, когда ранг её основной матрицы равен рангу её расширенной матрицы.