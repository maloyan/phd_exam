\section{
    Мера множества. Измеримые функции. Интеграл Лебега и его основные свойства.
}

1. \textbf{Мера множества}

Пусть задано множество ${\displaystyle X}$ с некоторым выделенным классом подмножеств 
${\displaystyle {\mathcal {F}}}$

Функция ${\displaystyle \mu \colon {\mathcal {F}}\to [0,\;\infty ]}$ называется мерой 
(иногда объёмом), если она удовлетворяет следующим аксиомам:
\begin{enumerate}
    \item ${\displaystyle \mu (\varnothing )=0}$ — мера пустого множества равна нулю;
    \item Для любых непересекающихся множеств ${\displaystyle A,B\in {\mathcal {F}},} {\displaystyle A\cap B=\varnothing }$
    
    $\mu (A\cup B)=\mu (A)+\mu (B)$ - мера объединения непересекающихся множеств равна сумме мер этих множеств (аддитивность, конечная аддитивность).
\end{enumerate}

2. \textbf{Измеримое множество}

Пусть ${\displaystyle (X,{\mathcal {F}})}$ и 
${\displaystyle (Y,{\mathcal {G}})}$ — два множества с выделенными алгебрами подмножеств. 

Тогда функция ${\displaystyle f:X\to Y}$ называется измеримой, если прообраз любого множества из ${\displaystyle {\mathcal {G}}}$ принадлежит 
${\displaystyle {\mathcal {F}}}$, то есть
${\displaystyle \forall B\in {\mathcal {G}},\;f^{-1}(B)\in {\mathcal {F}}}$

3. \textbf{Интеграл Лебега}

Дано пространство с мерой ${\displaystyle (X,{\mathcal {F}},\mu )}$, 
и на нём определена измеримая функция ${\displaystyle f\colon (X,{\mathcal {F}})\to (\mathbb {R} ,{\mathcal {B}}(\mathbb {R} ))}$.

Пусть ${\displaystyle f}$ — произвольная измеримая функция,  $A\in {\mathcal  {F}}$ 
произвольное измеримое множество. Тогда по определению
${\displaystyle \int \limits _{A}f(x)\,\mu (dx)=\int \limits _{X}f(x)\,\mathbf {1} _{A}(x)\,\mu (dx)}$, 
где ${\displaystyle \mathbf {1} _{A}(x)}$ — индикатор-функция множества ${\displaystyle A}$.

