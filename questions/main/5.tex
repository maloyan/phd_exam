\section{
    Абсолютная и условная сходимость ряда. \\
    Свойства абсолютно сходящихся рядов. \\
    Перестановка членов ряда. Теорема Римана. Умножение рядов. 
}

1. \textbf{Абсолютная сходимость}

Сходящийся ряд ${\displaystyle \sum a_{n}}$ называется сходящимся абсолютно, 
если сходится ряд из модулей ${\displaystyle \sum |a_{n}|}$, 
иначе — сходящимся условно.

2. \textbf{Условная сходимость}

Ряд ${\displaystyle \sum _{n=0}^{\infty }a_{n}}$ называется условно сходящимся, 
если ${\displaystyle \lim _{m\to \infty }\sum _{n=0}^{m}a_{n}}$ существует 
(и не бесконечен), но ${\displaystyle \sum _{n=0}^{\infty }|a_{n}|=\infty }$

3. Свойства
\begin{enumerate}
    \item Если ряд условно сходится, то ряды, 
    составленные из его положительных и отрицательных членов, расходятся.
    \item Путём изменения порядка членов условно сходящегося ряда можно получить ряд, 
    сходящийся к любой наперёд заданной сумме или же расходящийся (\textbf{теорема Римана}).
    \item При почленном умножении двух условно сходящихся рядов может 
    получиться расходящийся ряд.
\end{enumerate}

