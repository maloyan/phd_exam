\section{
	Непрерывные функции одной переменной и их свойства. \\
	Равномерная непрерывность. Равностепенная непрерывность семейства функций. Теорема Арцела.
}

1. Функция $f$ \textbf{непрерывна} в точке $x_{0}$, предельной для множества $D$, если $f$ имеет предел в точке $x_{0}$, и этот предел совпадает со значением функции $f(x_{0})$:

$$
\lim _{x\rightarrow x_{0}} f(x) = f(x_{0}) 
$$

2. Числовая функция вещественного переменного $f$ \textbf{равномерно} непрерывна на множестве $M$, если: 
$$
\forall \varepsilon >0 \colon 
~\exists \delta =\delta (\varepsilon )>0\colon 
~\forall x_{1},x_{2}\in M\colon 
~|x_{1}-x_{2}|<\delta \Rightarrow |f(x_{1})-f(x_{2})|<\varepsilon
$$

3. Семейство функций $D$ называется \textbf{равностепенно непрерывным} на данном отрезке $[a, b]$, если 
$$
\forall \varepsilon >0 \colon 
~\exists \delta =\delta (\varepsilon )>0\colon
~\forall f \in D, \forall x_{1},x_{2} \in[a, b]\colon
~|x_{1}-x_{2}| < \delta \Rightarrow |f(x_{1})-f(x_{2})|<\varepsilon
$$

4. \textbf{Теорема Арцела-Асколи}

Множество $D$ в семействе непрерывных функций $D \subset C[a,\;b]$ компактно (подмножество множества сходится к элементу данного множества)$\Leftrightarrow$
\begin{enumerate}
	\item $D$ замкнуто
	\item $D$ равномерно ограничено (все элементы этого множества ограничены)
	\item $D$ равностепенно непрерывно
\end{enumerate}