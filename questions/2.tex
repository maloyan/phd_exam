\section{
	Функции многих переменных. Полный дифференциал, и его геометрический смысл. \\
    Достаточные условия дифференцируемости. Градиент. 
}

1. Полный дифференциал

\textbf{Полным приращением} $dz$ функции $z=f(x,\;y)$, называется 
$$dz=f'_{x}(x, y)\varDelta x + f'_{y}(x,y)\varDelta y + \alpha \varDelta x + \beta \varDelta y$$
где $\alpha$ и $\beta$ – бесконечно малые функции при $\varDelta x \rightarrow 0,\; \varDelta y \rightarrow 0$

\textbf{Полным дифференциалом} $dz$ функции $z=f(x,\;y)$,
дифференцируемой в точке $(x,\;y)$, называется главная часть ее полного
приращения в этой точке, линейная относительно приращений аргументов $x$ и 
$y$, то есть $dz=f'_{x}(x, y)\varDelta  x + f'_{y}(x,y)\varDelta  y$

2. Достаточное условие дифференцируемости

Для того, чтобы функция $f(x)$ была дифференцируема в точке $x_0$ необходимо и достаточно, 
чтобы у нее существовала производная в этой точке. При этом
$$
\varDelta y = f(x_0 + \varDelta x) - f(x_0) = f'(x_0)\varDelta x + \alpha \varDelta x
$$

3. Градиент

Для случая трёхмерного пространства градиентом дифференцируемой в некоторой области скалярной функции 
$\varphi =\varphi (x,y,z)$ называется векторная функция с компонентами

$$
\frac{\partial \varphi }{\partial x},
\frac{\partial \varphi }{\partial y},
\frac{\partial \varphi }{\partial z}
$$