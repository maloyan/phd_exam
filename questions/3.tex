\section{
    Определенный интеграл. Интегрируемость непрерывной функции. \\
    Первообразная непрерывной функции.  \\
    Приближенное вычисление определенных интегралов. \\
    Формулы трапеций и Симпсона, оценки погрешностей. \\
    Понятие о методе Гаусса. 
}

1. Определенный интеграл

Пусть функция $f(x)$ определена на отрезке $[a;b]$. Разобьём $[a;b]$ на части несколькими произвольными точками: 
$a=x_{{0}}<x_{{1}}<x_{{2}}<\ldots <x_{{n}}=b$. Тогда говорят, что произведено разбиение $R$ отрезка $[a;b]$. 
Далее, для каждого $i$ от $0$ до $n-1$ выберем произвольную точку $\xi _{{i}}\in [x_{{i}};x_{{i+1}}]$.

\textbf{Определённым интегралом} от функции $f(x)$ на отрезке $[a;b]$ называется предел интегральных сумм при стремлении 
ранга разбиения к нулю $\lambda _{{R}}\rightarrow 0$, если он существует независимо от разбиения 
$R$ и выбора точек $\xi _{{i}}$, то есть

$$
\int \limits _{{a}}^{{b}}f(x)dx=\lim \limits _{{\Delta x\rightarrow 0}}\sum \limits _{{i=0}}^{{n-1}}f(\xi _{{i}})\Delta x_{{i}}
$$

Если существует указанный предел, то функция $f(x)$ называется \textbf{интегрируемой} на $[a;b]$ по Риману.

2. Первообразная

\textbf{Первообразной} для данной функции $f(x)$ называют
такую функцию $F(x)$, производная которой равна $f$ (на всей области определения $f$), то есть 
$F'(x)=f(x)$.

3. Приближенное вычисление интеграла

\begin{enumerate}
    \item Формула трапеций 
    $\displaystyle \int_{a}^{b}f(x)\,dx\approx \sum _{{i=0}}^{{n-1}}{\frac  {f(x_{i})+f(x_{{i+1}})}{2}}(x_{{i+1}}-x_{{i}})$.
    
    Погрешность ${\displaystyle \left|E(f)\right|\leqslant {\frac {\left(b-a\right)^{3}}{12n^{2}}}\max _{x\in [a,b]}\left|f''(x)\right|,{\frac {(b-a)^{3}}{12n^{2}}}={\frac {nh^{3}}{12}}.}$
    \item Формула Симпсона 
    $\displaystyle \int \limits _{a}^{b}f(x)dx \approx \frac{b-a}{6} \left(f(a)+4f\left(\frac{a+b}{2}\right)+f(b)\right)$
    
    Погрешность $\displaystyle E(f)=-{\frac  {(b-a)^{5}}{2880}}{{f^{{(4)}}(\zeta )}},\ \ \ \zeta \in [a,b].$
    
    \item метод Гаусса: узлы интегрирования  на отрезке  располагаются не равномерно, а выбираются таким образом,
    чтобы при наименьшем возможном числе узлов точно интегрировать многочлены наивысшей возможной степени.
\end{enumerate}