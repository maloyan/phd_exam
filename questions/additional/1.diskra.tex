\section{Элементы теории алгоритмов, математической логики и дискретной математики}

\subsection{Понятие алгоритма и его уточнения. Машины Тьюринга, нормальные алгоритмы Маркова; эквивалентность этих уточнений. Тезис Тьюринга. Понятие алгоритмической неразрешимости; примеры алгоритмически неразрешимых проблем.}
\subsection{Временная и пространственная алгебраическая сложность алгоритма в худшем случае и в среднем; символы $\mathbf{O}$, $\Omega$, и $\Theta$ в асимптотических оценках сложности. Сложность в худшем случае сортировки слияниями, быстрой сортировки; сложность в среднем быстрой сортировки. Понятие битовой сложности. Нижние границы сложности класса алгоритмов. Нижние границы сложности классов алгоритмов сортировки и поиска с помощью сравнений. Алгоритм, оптимальный в данном классе, оптимальность по порядку сложности.}
\subsection{Классы Р и NP алгоритмов распознавания языков. Проблема P $\stackrel{?}{=}$ NP . Полиномиальная сводимость; NP -полные задачи (формулировка основных фактов, примеры).}
\subsection{Логика 1-го порядка. Выполнимость и общезначимость. Общая схема метода резолюций. Логические программы. SLD-резолютивные вычисления логических программ. Правильные и вычислимые ответы на запросы к логическим программам. Стандартная стратегия выполнения логических программ.}
\subsection{Теорема Поста о полноте систем функций в алгебре логики. Графы, деревья, планарные графы, их свойства. }