\section{
    Ортогональные преобразования в евклидовом пространстве \\
    и ортогональные матрицы. Свойства ортогональных матриц.
}

1. \textbf{Ортогональное преобразование} — линейное преобразование 
${\displaystyle A}$ евклидова пространства ${\displaystyle L}$, 
сохраняющее длины или (что эквивалентно) скалярное произведение векторов. 
Это означает, что для любых двух векторов ${\displaystyle x,y\in L}$ выполняется равенство

${\displaystyle \langle A(x),\,A(y)\rangle =\langle x,\,y\rangle}$

2. \textbf{Ортогональная матрица} $A$ такая, что $AA^{{T}}=A^{{T}}A=E$

3. Свойства
\begin{itemize}
    \item Ортогональная матрица является унитарной 
    ${\displaystyle Q^{-1}=Q^{*}}$ и, следовательно, нормальной 
    ${\displaystyle Q^{*}Q=QQ^{*}}$
    \item Cкалярное произведение строки на саму себя равно 1, а на любую другую строку — 0. Это же справедливо и для столбцов.
    
    \item Определитель ортогональной матрицы равен ${\displaystyle \pm 1}$, 
    \item Множество ортогональных матриц порядка ${\displaystyle n}$ 
    над полем ${\displaystyle k}$ образует группу по умножению, 
    так называемую ортогональную группу которая обозначается ${\displaystyle O_{n}(k)}$ 
    или ${\displaystyle O(n,\;k)}$ (если ${\displaystyle k}$ опускается, 
    то предполагается ${\displaystyle k=\mathbb {R} }$.
    \item Линейный оператор, заданный ортогональной матрицей, 
    переводит ортонормированный базис линейного пространства в ортонормированный.
    \item Матрица вращения является специальной ортогональной (квадратная, оперделитель 1). 
    Матрица отражения является ортогональной.
    \item Любая вещественная ортогональная матрица подобна 
    (одного порядка и существует P того же порядка, такая что: ${\displaystyle B=P^{-1}AP}$)
    блочно-диагональной матрице с блоками вида $(\pm 1)$ и 
    ${\displaystyle {\begin{pmatrix}\ \ \ \cos \varphi &\sin \varphi \\-\sin \varphi &\cos \varphi \end{pmatrix}}.}$
\end{itemize}