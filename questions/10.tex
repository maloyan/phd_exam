\section{
    Функции комплексного переменного. Условия Коши-Римана. \\
    Геометрический смысл аргумента и модуля производной. 
}

1. Комплексные функции

Каждая комплексная функция ${\displaystyle w=f(z)=f(x+iy)}$
может рассматриваться как пара вещественных функций от двух переменных: 
${\displaystyle f(z)=u(x,y)+iv(x,y),}$ определяющих её вещественную и мнимую часть соответственно. 
Функции ${\displaystyle u,v}$ называются компонентами комплексной функции ${\displaystyle f(z)}$.

2. \textbf{Условия Коши-Римана}

Комплексная функция дифференцируема, если выполняется условия Коши-Римана:
$$
\frac{\partial u}{\partial x} = \frac{\partial v}{\partial y};
\frac{\partial u}{\partial y} = -\frac{\partial v}{\partial x}
$$

3. Аргумент и модуль производной
\begin{itemize}
    \item модуль определяет коэффициент маштабирования и определяет расстояние между точками в окрестности точки ${\displaystyle z}$
    \item аргумент определяет угол поворота гладкой кривой, проходящей через данную точку ${\displaystyle z}$
\end{itemize}

