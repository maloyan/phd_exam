\section{
	Числовые ряды. Сходимость рядов. Критерий Коши. \\
    Достаточные признаки сходимости (Коши, Деламбера, интегральный, Лейбница). 
}

1. Числовой ряд - беконечная сумма $\displaystyle \sum _{{i=1}}^{{\infty }}a_{i}$. Частичная сумма $\displaystyle S_n  \sum _{{i=1}}^{{n}}a_{i}$. 

2. Если последовательность частичных сумм имеет предел ${\displaystyle S}$ (конечный или бесконечный), 
то говорят, что сумма ряда равна ${\displaystyle S.}$ При этом, если предел конечен, то говорят, что ряд сходится. 
Если предел не существует или бесконечен, то говорят, что ряд расходится.

3. Критерий Коши

Последовательность ${\displaystyle x_{n}}$ называется последовательностью Коши или \textbf{фундаментальной}, если 
$${\displaystyle \forall \varepsilon >0,\exists N(\varepsilon ),\forall n,m>N(\varepsilon ):\mid x_{n}-x_{m}\mid <\varepsilon }$$

\textbf{Теорема (Критерий Коши)}. Для того, чтобы последовательность ${\displaystyle \{{x_{n}\}}}$ сходилась, 
необходимо и достаточно чтобы она была фундаментальной.

4. \textbf{Признак Д'Аламбера}
\begin{itemize}
    \item Ряд абсолютно сходится, если начиная с некоторого номера, выполняется неравенство
    ${\displaystyle \left|{\frac {a_{n+1}}{a_{n}}}\right|\leqslant q,~0<q<1}$
    \item Ряд расходится, если ${\displaystyle \left|{\frac {a_{n+1}}{a_{n}}}\right|\geqslant 1}$
    \item Если же, начиная с некоторого номера, 
    ${\displaystyle \left|{\frac {a_{n+1}}{a_{n}}}\right|<1}$, при этом не существует такого ${\displaystyle q}$ , 
    ${\displaystyle 0<q<1}$, что ${\displaystyle \left|{\frac {a_{n+1}}{a_{n}}}\right|\leqslant q}$ для всех ${\displaystyle n}$, 
    начиная с некоторого номера, то в этом случае ряд может как сходиться, так и расходиться.
\end{itemize}

5. \textbf{Признак Коши}
\begin{itemize}
    \item Ряд сходится, если начиная с некоторого номера, выполняется неравенство
    $\displaystyle \sqrt[{n}]{a_{n}} \leq q,~0<q<1$
    \item Ряд расходится, если ${\displaystyle {\sqrt[{n}]{a_{n}}}>1}$
    \item Если ${\displaystyle {\sqrt[{n}]{a_{n}}}=1}$, то это сомнительный случай и необходимы дополнительные исследования.
    \item Если же, начиная с некоторого номера, 
    ${\displaystyle {\sqrt[{n}]{a_{n}}}<1}$, при этом не существует такого ${\displaystyle q}$ , 
    ${\displaystyle 0<q<1}$, что $ {\displaystyle {\sqrt[{n}]{a_{n}}}\leqslant q}$ для всех ${\displaystyle n}$, 
    начиная с некоторого номера, то в этом случае ряд может как сходиться, так и расходиться.
\end{itemize}

6. \textbf{Признак Лейбница}

Пусть дан знакочередующийся ряд ${\displaystyle S=\sum _{n=1}^{\infty }(-1)^{n-1}b_{n},\ b_{n}\geq 0}$, для которого выполняются следующие условия:
\begin{enumerate}
    \item ${\displaystyle b_{n}\geq b_{n+1}}$, начиная с некоторого номера ${\displaystyle n\geq N}$,
    \item ${\displaystyle \lim _{n\to \infty }b_{n}=0.}$
\end{enumerate}
Тогда такой ряд сходится.

