\section{
    Теорема Коши об интеграле по замкнутому контуру. \\
    Интеграл Коши. Ряд Тейлора.
}

1. \textbf{Интегральная Теорема Коши}

Пусть ${\displaystyle D\subset \mathbb {C} }$ -- область, комплексная функция ${\displaystyle f(z)}$.
Тогда, если 
\begin{itemize}
    \item ${\displaystyle f(z)}$ голоморфна в ${\displaystyle D}$ и непрерывна в замыкании ${\displaystyle {\overline {D}}}$. 
    \item ${\displaystyle \Gamma}$ без петель
    \item ${\displaystyle \Gamma}$ конечна и без углов
\end{itemize}   
 справедливо соотношение 
${\displaystyle \oint \limits _{\Gamma }\,f(z)\,dz=0}$

2. \textbf{Интеграл Коши}

Пусть ${\displaystyle D}$ — область на комплексной плоскости с кусочно-гладкой границей 
${\displaystyle \Gamma =\partial D}$, функция ${\displaystyle f(z)}$ голоморфна в 
${\displaystyle {\overline {D}}}$, и ${\displaystyle z_{0}}$ — точка внутри области 
${\displaystyle D}$. Тогда справедлива следующая формула Коши:

${\displaystyle f(z_{0})={\frac {1}{2\pi i}}\int \limits _{\Gamma }{\frac {f(z)}{z-z_{0}}}\,dz}$

3. \textbf{Ряд Тейлора}

Рядом Тейлора в точке ${\displaystyle a}$ функции ${\displaystyle f(z)}$ комплексной переменной 
${\displaystyle z}$, удовлетворяющей в некоторой окрестности ${\displaystyle U\subseteq \mathbb {C}}$
точки ${\displaystyle a}$ условиям Коши — Римана, называется степенной Ряд

${\displaystyle f(z)=\sum _{n=0}^{+\infty }{\frac {f^{(n)}(a)}{n!}}(z-a)^{n}}.$