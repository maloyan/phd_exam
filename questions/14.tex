\section{
    Линейные преобразования. Квадратичные формы. \\
    Приведение их к каноническому виду линейными преобразованиями в комплексной и \\
    действительной областях. Закон инерции.
}

1. \textbf{Линейное отображение}

Линейным отображением векторного пространства ${\displaystyle V}$ 
над полем ${\displaystyle K}$ в векторное пространство 
${\displaystyle W}$ над тем же полем ${\displaystyle K}$ 
(линейным оператором из ${\displaystyle V}$ в ${\displaystyle W}$) 
называется \textbf{отображение} ${\displaystyle f\colon V\to W}$, удовлетворяющее условию линейности:

\begin{itemize}
    \item ${\displaystyle f(x+y)=f(x)+f(y)}$,
    \item ${\displaystyle f(\alpha x)=\alpha f(x)}$.
    
    для всех ${\displaystyle x,y\in V}$ и ${\displaystyle \alpha \in K}$.
\end{itemize}

Если ${\displaystyle V}$ и ${\displaystyle W}$ — это одно и то же векторное пространство, 
то ${\displaystyle f}$ — не просто линейное отображение, а \textbf{линейное преобразование}.

2. Пусть ${\displaystyle L}$ есть векторное пространство над полем 
${\displaystyle K}$ и ${\displaystyle e_{1},e_{2},\dots ,e_{n}}$ — базис в ${\displaystyle L}$.
Функция ${\displaystyle Q:L\to K}$ называется \textbf{квадратичной формой}, если её можно представить в виде:

${\displaystyle Q(x)=\sum _{i,j=1}^{n}a_{ij}x_{i}x_{j},}$ где 
${\displaystyle x=x_{1}e_{1}+x_{2}e_{2}+\cdots +x_{n}e_{n}}$, а 
${\displaystyle a_{ij}}$ — некоторые элементы поля ${\displaystyle K}$.

3. Метод Лагранжа приведения к каноническому виду

Каннонический вид: $\displaystyle f(X) = \lambda_1x_1^2 + \lambda_2x_2^2 + ... + \lambda_nx_n^2$

Пошаговое преобразование к канноническому виду методом Лагранжа:

Для $x_1$:

$f(x_{1},x_{2},...,x_{n})=(a_{{11}}x_{1}^{2}+2a_{{12}}x_{1}x_{2}+...+2a_{{1n}}x_{1}x_{n})+f_{1}(x_{2},x_{3},...,x_{n})= \\
{\displaystyle ={\frac {1}{a_{11}}}(a_{11}x_{1}+a_{12}x_{2}+...+a_{1n}x_{n})^{2}-{\frac {1}{a_{11}}}(a_{12}x_{2}+...+a_{1n}x_{n})^{2}+f_{1}(x_{2},x_{3},...,x_{n})=} \\
={\frac  {1}{a_{{11}}}}(a_{{11}}x_{1}+a_{{12}}x_{2}+...+a_{{1n}}x_{n})^{2}-{\frac  {1}{a_{{11}}}}(a_{{12}}x_{2}+...+a_{{1n}}x_{n})^{2}+f_{1}(x_{2},x_{3},...,x_{n})= \\
{\displaystyle ={\frac {1}{a_{11}}}y_{1}^{2}+f_{2}(x_{2},x_{3},...,x_{n})}={\frac  {1}{a_{{11}}}}y_{1}^{2}+f_{2}(x_{2},x_{3},...,x_{n})$

4. \textbf{Закон инерции}

Число положительных, отрицательных и нулевых канонических коэфициентов квадратичной формы 
не зависит от преобразования, с помощью которого квадатичная форма приводится к каноническому виду