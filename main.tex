\documentclass[conference]{./styles/acmsiggraph}
\usepackage{comment} % enables the use of multi-line comments (\ifx \fi)
\usepackage{lipsum} %This package just generates Lorem Ipsum filler text.
\usepackage{fullpage} % changes the margin
\usepackage{enumitem} % for customizing enumerate tags
\usepackage{amsmath,amsthm,amssymb}
\usepackage{listings}
\usepackage{graphicx}
\usepackage{etoolbox}   % for booleans and much more
\usepackage{verbatim}   % for the comment environment
\usepackage[dvipsnames]{xcolor}
\usepackage{fancyvrb}
\usepackage{hyperref}
\usepackage{menukeys}
\usepackage{titlesec}

\usepackage[T2A]{fontenc}
\usepackage[utf8]{inputenc}

\setlength{\parskip}{.8mm}
\setcounter{MaxMatrixCols}{20}

\title{\huge Экзамен \\ {\LARGE Ответы на вопросы}}
\author{\Large Student Name \\ student@uw.edu}
\pdfauthor{Student Name}

\hypersetup{
	colorlinks=true,
	urlcolor=[rgb]{0.97,0,0.30},
	anchorcolor={0.97,0,0.30},
	linkcolor=[rgb]{0.97,0,0.30},
	filecolor=[rgb]{0.97,0,0.30},
}

% redefine \VerbatimInput
\RecustomVerbatimCommand{\VerbatimInput}{VerbatimInput}%
{fontsize=\footnotesize,
 %
 frame=lines,  % top and bottom rule only
 framesep=2em, % separation between frame and text
 rulecolor=\color{Gray},
 %
 label=\fbox{\color{Black}\textbf{OUTPUT}},
 labelposition=topline,
 %
 commandchars=\|\(\), % escape character and argument delimiters for
                      % commands within the verbatim
 commentchar=*        % comment character
}

\titlespacing*{\section}{0pt}{5.5ex plus 1ex minus .2ex}{2ex}
\titlespacing*{\subsection}{0pt}{3ex}{2ex}

\setcounter{secnumdepth}{4}
\renewcommand\theparagraph{\thesubsubsection.\arabic{paragraph}}
\newcommand\subsubsubsection{\paragraph}

\setlength{\parskip}{0.5em}

% a macro for hiding answers
\newbool{hideanswers}
\setbool{hideanswers}{false}
\newenvironment{answer}{}{}
\ifbool{hideanswers}{\AtBeginEnvironment{answer}{\comment} %
\AtEndEnvironment{answer}{\endcomment}}{}

\newcommand{\points}[1]{\hfill \normalfont{(\textit{#1pts})}}
\newcommand{\pointsin}[1]{\normalfont{(\textit{#1pts})}}

\begin{document}
\section{
	Непрерывные функции одной переменной и их свойства. \\
	Равномерная непрерывность. Равностепенная непрерывность семейства функций. Теорема Арцела.
}

1. Функция $f$ \textbf{непрерывна} в точке $x_{0}$, предельной для множества $D$, если $f$ имеет предел в точке $x_{0}$, и этот предел совпадает со значением функции $f(x_{0})$:

$$
\lim _{x\rightarrow x_{0}} f(x) = f(x_{0}) 
$$

2. Числовая функция вещественного переменного $f$ \textbf{равномерно} непрерывна на множестве $M$, если: 
$$
\forall \varepsilon >0 \colon 
~\exists \delta =\delta (\varepsilon )>0\colon 
~\forall x_{1},x_{2}\in M\colon 
~|x_{1}-x_{2}|<\delta \Rightarrow |f(x_{1})-f(x_{2})|<\varepsilon
$$

3. Семейство функций $D$ называется \textbf{равностепенно непрерывным} на данном отрезке $[a, b]$, если 
$$
\forall \varepsilon >0 \colon 
~\exists \delta =\delta (\varepsilon )>0\colon
~\forall f \in D, \forall x_{1},x_{2} \in[a, b]\colon
~|x_{1}-x_{2}| < \delta \Rightarrow |f(x_{1})-f(x_{2})|<\varepsilon
$$

4. \textbf{Теорема Арцела-Асколи}

Множество $D$ в семействе непрерывных функций $D \subset C[a,\;b]$ компактно (подмножество множества сходится к элементу данного множества)$\Leftrightarrow$
\begin{enumerate}
	\item $D$ замкнуто
	\item $D$ равномерно ограничено (все элементы этого множества ограничены)
	\item $D$ равностепенно непрерывно
\end{enumerate}
\section{
	Функции многих переменных. Полный дифференциал, и его геометрический смысл. \\
    Достаточные условия дифференцируемости. Градиент. 
}

1. Полный дифференциал

\textbf{Полным приращением} $dz$ функции $z=f(x,\;y)$, называется 
$$dz=f'_{x}(x, y)\varDelta x + f'_{y}(x,y)\varDelta y + \alpha \varDelta x + \beta \varDelta y$$
где $\alpha$ и $\beta$ – бесконечно малые функции при $\varDelta x \rightarrow 0,\; \varDelta y \rightarrow 0$

\textbf{Полным дифференциалом} $dz$ функции $z=f(x,\;y)$,
дифференцируемой в точке $(x,\;y)$, называется главная часть ее полного
приращения в этой точке, линейная относительно приращений аргументов $x$ и 
$y$, то есть $dz=f'_{x}(x, y)\varDelta  x + f'_{y}(x,y)\varDelta  y$

2. Достаточное условие дифференцируемости

Для того, чтобы функция $f(x)$ была дифференцируема в точке $x_0$ необходимо и достаточно, 
чтобы у нее существовала производная в этой точке. При этом
$$
\varDelta y = f(x_0 + \varDelta x) - f(x_0) = f'(x_0)\varDelta x + \alpha \varDelta x
$$

3. Градиент

Для случая трёхмерного пространства градиентом дифференцируемой в некоторой области скалярной функции 
$\varphi =\varphi (x,y,z)$ называется векторная функция с компонентами

$$
\frac{\partial \varphi }{\partial x},
\frac{\partial \varphi }{\partial y},
\frac{\partial \varphi }{\partial z}
$$
\section{
    Определенный интеграл. Интегрируемость непрерывной функции. \\
    Первообразная непрерывной функции.  \\
    Приближенное вычисление определенных интегралов. \\
    Формулы трапеций и Симпсона, оценки погрешностей. \\
    Понятие о методе Гаусса. 
}

1. Определенный интеграл

Пусть функция $f(x)$ определена на отрезке $[a;b]$. Разобьём $[a;b]$ на части несколькими произвольными точками: 
$a=x_{{0}}<x_{{1}}<x_{{2}}<\ldots <x_{{n}}=b$. Тогда говорят, что произведено разбиение $R$ отрезка $[a;b]$. 
Далее, для каждого $i$ от $0$ до $n-1$ выберем произвольную точку $\xi _{{i}}\in [x_{{i}};x_{{i+1}}]$.

\textbf{Определённым интегралом} от функции $f(x)$ на отрезке $[a;b]$ называется предел интегральных сумм при стремлении 
ранга разбиения к нулю $\lambda _{{R}}\rightarrow 0$, если он существует независимо от разбиения 
$R$ и выбора точек $\xi _{{i}}$, то есть

$$
\int \limits _{{a}}^{{b}}f(x)dx=\lim \limits _{{\Delta x\rightarrow 0}}\sum \limits _{{i=0}}^{{n-1}}f(\xi _{{i}})\Delta x_{{i}}
$$

Если существует указанный предел, то функция $f(x)$ называется \textbf{интегрируемой} на $[a;b]$ по Риману.

2. Первообразная

\textbf{Первообразной} для данной функции $f(x)$ называют
такую функцию $F(x)$, производная которой равна $f$ (на всей области определения $f$), то есть 
$F'(x)=f(x)$.

3. Приближенное вычисление интеграла

\begin{enumerate}
    \item Формула трапеций 
    $\displaystyle \int_{a}^{b}f(x)\,dx\approx \sum _{{i=0}}^{{n-1}}{\frac  {f(x_{i})+f(x_{{i+1}})}{2}}(x_{{i+1}}-x_{{i}})$.
    
    Погрешность ${\displaystyle \left|E(f)\right|\leqslant {\frac {\left(b-a\right)^{3}}{12n^{2}}}\max _{x\in [a,b]}\left|f''(x)\right|,{\frac {(b-a)^{3}}{12n^{2}}}={\frac {nh^{3}}{12}}.}$
    \item Формула Симпсона 
    $\displaystyle \int \limits _{a}^{b}f(x)dx \approx \frac{b-a}{6} \left(f(a)+4f\left(\frac{a+b}{2}\right)+f(b)\right)$
    
    Погрешность $\displaystyle E(f)=-{\frac  {(b-a)^{5}}{2880}}{{f^{{(4)}}(\zeta )}},\ \ \ \zeta \in [a,b].$
    
    \item метод Гаусса: узлы интегрирования  на отрезке  располагаются не равномерно, а выбираются таким образом,
    чтобы при наименьшем возможном числе узлов точно интегрировать многочлены наивысшей возможной степени.
\end{enumerate}
\section{
	Числовые ряды. Сходимость рядов. Критерий Коши. \\
    Достаточные признаки сходимости (Коши, Деламбера, интегральный, Лейбница). 
}

1. Числовой ряд - беконечная сумма $\displaystyle \sum _{{i=1}}^{{\infty }}a_{i}$. Частичная сумма $\displaystyle S_n  \sum _{{i=1}}^{{n}}a_{i}$. 

2. Если последовательность частичных сумм имеет предел ${\displaystyle S}$ (конечный или бесконечный), 
то говорят, что сумма ряда равна ${\displaystyle S.}$ При этом, если предел конечен, то говорят, что ряд сходится. 
Если предел не существует или бесконечен, то говорят, что ряд расходится.

3. Критерий Коши

$
{\displaystyle \forall \,\varepsilon >0,\exists \,\nu _{\varepsilon },\forall n,\forall p,}
{\displaystyle n\geqslant \,\nu _{\varepsilon }\Rightarrow \left|\sum _{k=1}^{p}{a_{n+k}}\right|\leqslant \varepsilon .}
$

4. \textbf{Признак Д'Аламбера}
\begin{itemize}
    \item Ряд абсолютно сходится, если начиная с некоторого номера, выполняется неравенство
    ${\displaystyle \left|{\frac {a_{n+1}}{a_{n}}}\right|\leqslant q,~0<q<1}$
    \item Ряд расходится, если ${\displaystyle \left|{\frac {a_{n+1}}{a_{n}}}\right|\geqslant 1}$
    \item Если же, начиная с некоторого номера, 
    ${\displaystyle \left|{\frac {a_{n+1}}{a_{n}}}\right|<1}$, при этом не существует такого ${\displaystyle q}$ , 
    ${\displaystyle 0<q<1}$, что ${\displaystyle \left|{\frac {a_{n+1}}{a_{n}}}\right|\leqslant q}$ для всех ${\displaystyle n}$, 
    начиная с некоторого номера, то в этом случае ряд может как сходиться, так и расходиться.
\end{itemize}

5. \textbf{Признак Коши}
\begin{itemize}
    \item Ряд сходится, если начиная с некоторого номера, выполняется неравенство
    $\displaystyle \sqrt[{n}]{a_{n}} \leq q,~0<q<1$
    \item Ряд расходится, если ${\displaystyle {\sqrt[{n}]{a_{n}}}>1}$
    \item Если ${\displaystyle {\sqrt[{n}]{a_{n}}}=1}$, то это сомнительный случай и необходимы дополнительные исследования.
    \item Если же, начиная с некоторого номера, 
    ${\displaystyle {\sqrt[{n}]{a_{n}}}<1}$, при этом не существует такого ${\displaystyle q}$ , 
    ${\displaystyle 0<q<1}$, что $ {\displaystyle {\sqrt[{n}]{a_{n}}}\leqslant q}$ для всех ${\displaystyle n}$, 
    начиная с некоторого номера, то в этом случае ряд может как сходиться, так и расходиться.
\end{itemize}

6. \textbf{Признак Лейбница}

Пусть дан знакочередующийся ряд ${\displaystyle S=\sum _{n=1}^{\infty }(-1)^{n-1}b_{n},\ b_{n}\geq 0}$, для которого выполняются следующие условия:
\begin{enumerate}
    \item ${\displaystyle b_{n}\geq b_{n+1}}$, начиная с некоторого номера ${\displaystyle n\geq N}$,
    \item ${\displaystyle \lim _{n\to \infty }b_{n}=0.}$
\end{enumerate}
Тогда такой ряд сходится.


\section{
    Абсолютная и условная сходимость ряда. \\
    Свойства абсолютно сходящихся рядов. \\
    Перестановка членов ряда. Теорема Римана. Умножение рядов. 
}

1. \textbf{Абсолютная сходимость}

Сходящийся ряд ${\displaystyle \sum a_{n}}$ называется сходящимся абсолютно, 
если сходится ряд из модулей ${\displaystyle \sum |a_{n}|}$, 
иначе — сходящимся условно.

2. \textbf{Условная сходимость}

Ряд ${\displaystyle \sum _{n=0}^{\infty }a_{n}}$ называется условно сходящимся, 
если ${\displaystyle \lim _{m\to \infty }\sum _{n=0}^{m}a_{n}}$ существует 
(и не бесконечен), но ${\displaystyle \sum _{n=0}^{\infty }|a_{n}|=\infty }$

3. Свойства
\begin{enumerate}
    \item Если ряд условно сходится, то ряды, 
    составленные из его положительных и отрицательных членов, расходятся.
    \item Путём изменения порядка членов условно сходящегося ряда можно получить ряд, 
    сходящийся к любой наперёд заданной сумме или же расходящийся (\textbf{теорема Римана}).
    \item При почленном умножении двух условно сходящихся рядов может 
    получиться расходящийся ряд.
\end{enumerate}


\section{
    Ряды и последовательности функций. Равномерная сходимость. \\
    Признак Вейерштрасса. Свойства равномерно сходящихся рядов \\
    (непрерывность суммы, почленное интегрирование и дифференцирование). 
}

1. \textbf{Функциональные ряды и последовательности}

$\displaystyle \sum _{{k=1}}^{{\infty }}{u_{k}}(x)$ - функциональный ряд, каждый элемент явяется функцией

${\displaystyle \ {S_{n}}(x)=\sum _{k=1}^{n}{u_{k}}(x)}$— n-ная частичная сумма.

2. \textbf{Поточечная сходимость}

Функциональная последовательность ${\displaystyle \ {u_{k}}(x)}$
сходится поточечно к функции ${\displaystyle \ {u}(x)}$, если 
${\displaystyle \forall x\in E\;\;\;\exists \lim _{k\rightarrow \infty }\ {u_{k}}(x)=\ {u}(x)}$

3. \textbf{Равномерная сходимость}

Существует функция ${\displaystyle \ u(x):E\mapsto \mathbb {C} }$
такая, что: ${\displaystyle \ \sup \mid {u_{k}}(x)-u(x)\mid {\stackrel {k\rightarrow \infty }{\longrightarrow }}0,~~x\in E}$

Факт равномерной сходимости последовательности ${\displaystyle \ {u_{k}}(x)}$
 к функции ${\displaystyle \ u(x)}$ записывается: 
 ${\displaystyle \ {u_{k}}(x)\rightrightarrows u(x)}$

4. \textbf{Признак Вейерштрасса}

Пусть ${\displaystyle \exists a_{n} : \forall x\in X : |u_{n}(x)|<a_{n}}$, кроме того, 
ряд ${\displaystyle \sum _{n=1}^{\infty }a_{n}}$ сходится. 
Тогда ряд ${\displaystyle \sum _{n=1}^{\infty }u_{n}(x)}$ сходится 
на множестве ${\displaystyle X}$ абсолютно и равномерно.

5. \textbf{Свойства равномерно сходящихся рядов}
\begin{itemize}
    \item Если $\displaystyle \{f_n\}$ и $\displaystyle \{g_n\}$ равномерно сходятся на множестве, то и 
    $\displaystyle \{f_n + g_n\}$, и $\displaystyle \{\alpha f_n\},~\alpha \in \mathbb {R}$ 
    тоже равномерно сходятся на этом множестве
   
    \item Если \textbf{последовательность интегрируемых} по Риману (по Лебегу) функций 
    ${\displaystyle f_{n} \rightrightarrows f}$ \textbf{равномерно сходится} на отрезке ${\displaystyle [a,b]}$, 
    то эта функция $\displaystyle f$ \textbf{также интегрируема} по Риману (по Лебегу), и \\
    ${\displaystyle \forall x\in [a,b] :~\lim _{n\to \infty }\int \limits _{a}^{x}f_{n}(t)dt=\int \limits _{a}^{x}f(t)dt}$ 
    и сходимость последовательности функций
    ${\displaystyle \int \limits _{a}^{x}f_{n}(t)dt \rightrightarrows \int \limits _{a}^{x}f(t)dt}$
    на отрезке ${\displaystyle [a,b]}$.

    \item Если последовательность непрерывно дифференцируемых на отрезке ${\displaystyle [a,b]}$ 
    \textbf{функций сходится} ${\displaystyle \{f_{n}\} \rightarrow x_{0}}$, 
    a последовательность их \textbf{производных равномерно сходится} на ${\displaystyle [a,b]}$,
    то последовательность ${\displaystyle \{f_{n}\}}$ \textbf{также равномерно сходится} 
    на ${\displaystyle [a,b]}$, её предел является непрерывно дифференцируемой на этом отрезке функцией.
\end{itemize}

\end{document}
