\documentclass[conference]{styles/acmsiggraph}
\usepackage{comment} % enables the use of multi-line comments (\ifx \fi)
\usepackage{lipsum} %This package just generates Lorem Ipsum filler text.
\usepackage{fullpage} % changes the margin
\usepackage{enumitem} % for customizing enumerate tags
\usepackage{amsmath,amsthm,amssymb}
\usepackage{listings}
\usepackage{graphicx}
\usepackage{etoolbox}   % for booleans and much more
\usepackage{verbatim}   % for the comment environment
\usepackage[dvipsnames]{xcolor}
\usepackage{fancyvrb}
\usepackage{hyperref}
\usepackage{menukeys}
\usepackage{titlesec}

\usepackage[T2A]{fontenc}
\usepackage[utf8]{inputenc}

\setlength{\parskip}{.8mm}
\setcounter{MaxMatrixCols}{20}

\title{\huge Экзамен \\ {\LARGE Ответы на вопросы}}
\author{\Large Student Name \\ student@uw.edu}
\pdfauthor{Student Name}

\hypersetup{
	colorlinks=true,
	urlcolor=[rgb]{0.97,0,0.30},
	anchorcolor={0.97,0,0.30},
	linkcolor=[rgb]{0.97,0,0.30},
	filecolor=[rgb]{0.97,0,0.30},
}

% redefine \VerbatimInput
\RecustomVerbatimCommand{\VerbatimInput}{VerbatimInput}%
{fontsize=\footnotesize,
 %
 frame=lines,  % top and bottom rule only
 framesep=2em, % separation between frame and text
 rulecolor=\color{Gray},
 %
 label=\fbox{\color{Black}\textbf{OUTPUT}},
 labelposition=topline,
 %
 commandchars=\|\(\), % escape character and argument delimiters for
                      % commands within the verbatim
 commentchar=*        % comment character
}

\titlespacing*{\section}{0pt}{5.5ex plus 1ex minus .2ex}{2ex}
\titlespacing*{\subsection}{0pt}{3ex}{2ex}

\setcounter{secnumdepth}{4}
\renewcommand\theparagraph{\thesubsubsection.\arabic{paragraph}}
\newcommand\subsubsubsection{\paragraph}

\setlength{\parskip}{0.5em}

% a macro for hiding answers
\newbool{hideanswers}
\setbool{hideanswers}{false}
\newenvironment{answer}{}{}
\ifbool{hideanswers}{\AtBeginEnvironment{answer}{\comment} %
\AtEndEnvironment{answer}{\endcomment}}{}

\newcommand{\points}[1]{\hfill \normalfont{(\textit{#1pts})}}
\newcommand{\pointsin}[1]{\normalfont{(\textit{#1pts})}}

\begin{document}
\part{Основная часть}
\section{
	Непрерывные функции одной переменной и их свойства. \\
	Равномерная непрерывность. Равностепенная непрерывность семейства функций. Теорема Арцела.
}

1. Функция $f$ \textbf{непрерывна} в точке $x_{0}$, предельной для множества $D$, если $f$ имеет предел в точке $x_{0}$, и этот предел совпадает со значением функции $f(x_{0})$:

$$
\lim _{x\rightarrow x_{0}} f(x) = f(x_{0}) 
$$

2. Числовая функция вещественного переменного $f$ \textbf{равномерно} непрерывна на множестве $M$, если: 
$$
\forall \varepsilon >0 \colon 
~\exists \delta =\delta (\varepsilon )>0\colon 
~\forall x_{1},x_{2}\in M\colon 
~|x_{1}-x_{2}|<\delta \Rightarrow |f(x_{1})-f(x_{2})|<\varepsilon
$$

3. Семейство функций $D$ называется \textbf{равностепенно непрерывным} на данном отрезке $[a, b]$, если 
$$
\forall \varepsilon >0 \colon 
~\exists \delta =\delta (\varepsilon )>0\colon
~\forall f \in D, \forall x_{1},x_{2} \in[a, b]\colon
~|x_{1}-x_{2}| < \delta \Rightarrow |f(x_{1})-f(x_{2})|<\varepsilon
$$

4. \textbf{Теорема Арцела-Асколи}

Множество $D$ в семействе непрерывных функций $D \subset C[a,\;b]$ компактно (подмножество множества сходится к элементу данного множества)$\Leftrightarrow$
\begin{enumerate}
	\item $D$ замкнуто
	\item $D$ равномерно ограничено (все элементы этого множества ограничены)
	\item $D$ равностепенно непрерывно
\end{enumerate}
\section{
	Функции многих переменных. Полный дифференциал, и его геометрический смысл. \\
    Достаточные условия дифференцируемости. Градиент. 
}

1. Полный дифференциал

\textbf{Полным приращением} $dz$ функции $z=f(x,\;y)$, называется 
$$dz=f'_{x}(x, y)\varDelta x + f'_{y}(x,y)\varDelta y + \alpha \varDelta x + \beta \varDelta y$$
где $\alpha$ и $\beta$ – бесконечно малые функции при $\varDelta x \rightarrow 0,\; \varDelta y \rightarrow 0$

\textbf{Полным дифференциалом} $dz$ функции $z=f(x,\;y)$,
дифференцируемой в точке $(x,\;y)$, называется главная часть ее полного
приращения в этой точке, линейная относительно приращений аргументов $x$ и 
$y$, то есть $dz=f'_{x}(x, y)\varDelta  x + f'_{y}(x,y)\varDelta  y$

2. Достаточное условие дифференцируемости

Для того, чтобы функция $f(x)$ была дифференцируема в точке $x_0$ 
достаточно непрерывности частных производных в этой точке.

3. Градиент

Для случая трёхмерного пространства градиентом дифференцируемой в некоторой области скалярной функции 
$\varphi =\varphi (x,y,z)$ называется векторная функция с компонентами

$$
\frac{\partial \varphi }{\partial x},
\frac{\partial \varphi }{\partial y},
\frac{\partial \varphi }{\partial z}
$$
\section{
    Определенный интеграл. Интегрируемость непрерывной функции. \\
    Первообразная непрерывной функции.  \\
    Приближенное вычисление определенных интегралов. \\
    Формулы трапеций и Симпсона, оценки погрешностей. \\
    Понятие о методе Гаусса. 
}

1. Определенный интеграл

Пусть функция $f(x)$ определена на отрезке $[a;b]$. Разобьём $[a;b]$ на части несколькими произвольными точками: 
$a=x_{{0}}<x_{{1}}<x_{{2}}<\ldots <x_{{n}}=b$. Тогда говорят, что произведено разбиение $R$ отрезка $[a;b]$. 
Далее, для каждого $i$ от $0$ до $n-1$ выберем произвольную точку $\xi _{{i}}\in [x_{{i}};x_{{i+1}}]$.

\textbf{Определённым интегралом} от функции $f(x)$ на отрезке $[a;b]$ называется предел интегральных сумм при стремлении 
ранга разбиения к нулю $\lambda _{{R}}\rightarrow 0$, если он существует независимо от разбиения 
$R$ и выбора точек $\xi _{{i}}$, то есть

$$
\int \limits _{{a}}^{{b}}f(x)dx=\lim \limits _{{\Delta x\rightarrow 0}}\sum \limits _{{i=0}}^{{n-1}}f(\xi _{{i}})\Delta x_{{i}}
$$

Если существует указанный предел, то функция $f(x)$ называется \textbf{интегрируемой} на $[a;b]$ по Риману.

2. Первообразная

\textbf{Первообразной} для данной функции $f(x)$ называют
такую функцию $F(x)$, производная которой равна $f$ (на всей области определения $f$), то есть 
$F'(x)=f(x)$.

3. Приближенное вычисление интеграла

\begin{enumerate}
    \item Формула трапеций 
    $\displaystyle \int_{a}^{b}f(x)\,dx\approx \sum _{{i=0}}^{{n-1}}{\frac  {f(x_{i})+f(x_{{i+1}})}{2}}(x_{{i+1}}-x_{{i}})$.
    
    Погрешность ${\displaystyle \left|E(f)\right|\leqslant {\frac {\left(b-a\right)^{3}}{12n^{2}}}\max _{x\in [a,b]}\left|f''(x)\right|,{\frac {(b-a)^{3}}{12n^{2}}}={\frac {nh^{3}}{12}}.}$
    \item Формула Симпсона 
    $\displaystyle \int \limits _{a}^{b}f(x)dx \approx \frac{b-a}{6} \left(f(a)+4f\left(\frac{a+b}{2}\right)+f(b)\right)$
    
    Погрешность $\displaystyle E(f)=-{\frac  {(b-a)^{5}}{2880}}{{f^{{(4)}}(\zeta )}},\ \ \ \zeta \in [a,b].$
    
    \item метод Гаусса: узлы интегрирования  на отрезке  располагаются не равномерно, а выбираются таким образом,
    чтобы при наименьшем возможном числе узлов точно интегрировать многочлены наивысшей возможной степени.
\end{enumerate}
\section{
	Числовые ряды. Сходимость рядов. Критерий Коши. \\
    Достаточные признаки сходимости (Коши, Деламбера, интегральный, Лейбница). 
}

1. Числовой ряд - беконечная сумма $\displaystyle \sum _{{i=1}}^{{\infty }}a_{i}$. Частичная сумма $\displaystyle S_n  \sum _{{i=1}}^{{n}}a_{i}$. 

2. Если последовательность частичных сумм имеет предел ${\displaystyle S}$ (конечный или бесконечный), 
то говорят, что сумма ряда равна ${\displaystyle S.}$ При этом, если предел конечен, то говорят, что ряд сходится. 
Если предел не существует или бесконечен, то говорят, что ряд расходится.

3. Критерий Коши

$
{\displaystyle \forall \,\varepsilon >0,\exists \,\nu _{\varepsilon },\forall n,\forall p,}
{\displaystyle n\geqslant \,\nu _{\varepsilon }\Rightarrow \left|\sum _{k=1}^{p}{a_{n+k}}\right|\leqslant \varepsilon .}
$

4. \textbf{Признак Д'Аламбера}
\begin{itemize}
    \item Ряд абсолютно сходится, если начиная с некоторого номера, выполняется неравенство
    ${\displaystyle \left|{\frac {a_{n+1}}{a_{n}}}\right|\leqslant q,~0<q<1}$
    \item Ряд расходится, если ${\displaystyle \left|{\frac {a_{n+1}}{a_{n}}}\right|\geqslant 1}$
    \item Если же, начиная с некоторого номера, 
    ${\displaystyle \left|{\frac {a_{n+1}}{a_{n}}}\right|<1}$, при этом не существует такого ${\displaystyle q}$ , 
    ${\displaystyle 0<q<1}$, что ${\displaystyle \left|{\frac {a_{n+1}}{a_{n}}}\right|\leqslant q}$ для всех ${\displaystyle n}$, 
    начиная с некоторого номера, то в этом случае ряд может как сходиться, так и расходиться.
\end{itemize}

5. \textbf{Признак Коши}
\begin{itemize}
    \item Ряд сходится, если начиная с некоторого номера, выполняется неравенство
    $\displaystyle \sqrt[{n}]{a_{n}} \leq q,~0<q<1$
    \item Ряд расходится, если ${\displaystyle {\sqrt[{n}]{a_{n}}}>1}$
    \item Если ${\displaystyle {\sqrt[{n}]{a_{n}}}=1}$, то это сомнительный случай и необходимы дополнительные исследования.
    \item Если же, начиная с некоторого номера, 
    ${\displaystyle {\sqrt[{n}]{a_{n}}}<1}$, при этом не существует такого ${\displaystyle q}$ , 
    ${\displaystyle 0<q<1}$, что $ {\displaystyle {\sqrt[{n}]{a_{n}}}\leqslant q}$ для всех ${\displaystyle n}$, 
    начиная с некоторого номера, то в этом случае ряд может как сходиться, так и расходиться.
\end{itemize}

6. \textbf{Признак Лейбница}

Пусть дан знакочередующийся ряд ${\displaystyle S=\sum _{n=1}^{\infty }(-1)^{n-1}b_{n},\ b_{n}\geq 0}$, для которого выполняются следующие условия:
\begin{enumerate}
    \item ${\displaystyle b_{n}\geq b_{n+1}}$, начиная с некоторого номера ${\displaystyle n\geq N}$,
    \item ${\displaystyle \lim _{n\to \infty }b_{n}=0.}$
\end{enumerate}
Тогда такой ряд сходится.


\section{
    Абсолютная и условная сходимость ряда. \\
    Свойства абсолютно сходящихся рядов. \\
    Перестановка членов ряда. Теорема Римана. Умножение рядов. 
}

1. \textbf{Абсолютная сходимость}

Сходящийся ряд ${\displaystyle \sum a_{n}}$ называется сходящимся абсолютно, 
если сходится ряд из модулей ${\displaystyle \sum |a_{n}|}$, 
иначе — сходящимся условно.

2. \textbf{Условная сходимость}

Ряд ${\displaystyle \sum _{n=0}^{\infty }a_{n}}$ называется условно сходящимся, 
если ${\displaystyle \lim _{m\to \infty }\sum _{n=0}^{m}a_{n}}$ существует 
(и не бесконечен), но ${\displaystyle \sum _{n=0}^{\infty }|a_{n}|=\infty }$

3. Свойства
\begin{enumerate}
    \item Если ряд условно сходится, то ряды, 
    составленные из его положительных и отрицательных членов, расходятся.
    \item Путём изменения порядка членов условно сходящегося ряда можно получить ряд, 
    сходящийся к любой наперёд заданной сумме или же расходящийся (\textbf{теорема Римана}).
    \item При почленном умножении двух условно сходящихся рядов может 
    получиться расходящийся ряд.
\end{enumerate}


\section{
    Ряды и последовательности функций. Равномерная сходимость. \\
    Признак Вейерштрасса. Свойства равномерно сходящихся рядов \\
    (непрерывность суммы, почленное интегрирование и дифференцирование). 
}

1. \textbf{Функциональные ряды и последовательности}

$\displaystyle \sum _{{k=1}}^{{\infty }}{u_{k}}(x)$ - функциональный ряд, каждый элемент явяется функцией

${\displaystyle \ {S_{n}}(x)=\sum _{k=1}^{n}{u_{k}}(x)}$— n-ная частичная сумма.

2. \textbf{Поточечная сходимость}

Функциональная последовательность ${\displaystyle \ {u_{k}}(x)}$
сходится поточечно к функции ${\displaystyle \ {u}(x)}$, если 
${\displaystyle \forall x\in E\;\;\;\exists \lim _{k\rightarrow \infty }\ {u_{k}}(x)=\ {u}(x)}$

3. \textbf{Равномерная сходимость}

Существует функция ${\displaystyle \ u(x):E\mapsto \mathbb {C} }$
такая, что: ${\displaystyle \ \sup \mid {u_{k}}(x)-u(x)\mid {\stackrel {k\rightarrow \infty }{\longrightarrow }}0,~~x\in E}$

Факт равномерной сходимости последовательности ${\displaystyle \ {u_{k}}(x)}$
 к функции ${\displaystyle \ u(x)}$ записывается: 
 ${\displaystyle \ {u_{k}}(x)\rightrightarrows u(x)}$

4. \textbf{Признак Вейерштрасса}

Пусть ${\displaystyle \exists a_{n} : \forall x\in X : |u_{n}(x)|<a_{n}}$, кроме того, 
ряд ${\displaystyle \sum _{n=1}^{\infty }a_{n}}$ сходится. 
Тогда ряд ${\displaystyle \sum _{n=1}^{\infty }u_{n}(x)}$ сходится 
на множестве ${\displaystyle X}$ абсолютно и равномерно.

5. \textbf{Свойства равномерно сходящихся рядов}
\begin{itemize}
    \item Если $\displaystyle \{f_n\}$ и $\displaystyle \{g_n\}$ равномерно сходятся на множестве, то и 
    $\displaystyle \{f_n + g_n\}$, и $\displaystyle \{\alpha f_n\},~\alpha \in \mathbb {R}$ 
    тоже равномерно сходятся на этом множестве
   
    \item Если \textbf{последовательность интегрируемых} по Риману (по Лебегу) функций 
    ${\displaystyle f_{n} \rightrightarrows f}$ \textbf{равномерно сходится} на отрезке ${\displaystyle [a,b]}$, 
    то эта функция $\displaystyle f$ \textbf{также интегрируема} по Риману (по Лебегу), и \\
    ${\displaystyle \forall x\in [a,b] :~\lim _{n\to \infty }\int \limits _{a}^{x}f_{n}(t)dt=\int \limits _{a}^{x}f(t)dt}$ 
    и сходимость последовательности функций
    ${\displaystyle \int \limits _{a}^{x}f_{n}(t)dt \rightrightarrows \int \limits _{a}^{x}f(t)dt}$
    на отрезке ${\displaystyle [a,b]}$.

    \item Если последовательность непрерывно дифференцируемых на отрезке ${\displaystyle [a,b]}$ 
    \textbf{функций сходится} ${\displaystyle \{f_{n}\} \rightarrow x_{0}}$, 
    a последовательность их \textbf{производных равномерно сходится} на ${\displaystyle [a,b]}$,
    то последовательность ${\displaystyle \{f_{n}\}}$ \textbf{также равномерно сходится} 
    на ${\displaystyle [a,b]}$, её предел является непрерывно дифференцируемой на этом отрезке функцией.
\end{itemize}
\section{
    Собственные и несобственные интегралы, зависящие от параметра. \\
    Равномерная сходимость по параметрам и ее признаки. \\
    Непрерывность, интегрирование и дифференцирование интегралов по параметру.
}

1. \textbf{Собственный интеграл}

Определенный интеграл называется собственным, если область интегрирования и функция интегрирования являются ограниченными

2. \textbf{Несобственный интеграл}

Определённый интеграл называется \textbf{несобственным}, 
если выполняется по крайней мере одно из следующих условий:
\begin{itemize}
    \item Область интегрирования является бесконечной ${\displaystyle [a,+\infty )}$.
    \item Функция ${\displaystyle f(x)}$ является неограниченной 
    в окрестности некоторых точек области интегрирования.
\end{itemize}

\textbf{Несобственный интеграл I рода} ${\displaystyle \exists \lim _{A\to +\infty }\int \limits _{a}^{A}f(x)dx=I\in \mathbb {R} }$

\textbf{Несобственный интеграл II рода} ${\displaystyle \forall \delta >0\Rightarrow \exists \int \limits _{a+\delta }^{b}f(x)dx={\mathcal {I}}(\delta )}$ или 
${\displaystyle \exists \int \limits _{a}^{b-\delta }f(x)dx={\mathcal {I}}(\delta )}$.
При этом
${\displaystyle \exists \lim _{\delta \to 0+0}{\mathcal {I}}(\delta )=I\in \mathbb {R} }$

3. \textbf{Интеграл, зависящий от параметра}

Пусть в двумерном евклидовом пространстве задана область ${\displaystyle {\overline {G}}=\left\{\left(x,y\right)|a\leq x\leq b,c\leq y\leq d\right\}}$
на которой определена функция ${\displaystyle f(x,y)}$ двух переменных.

Пусть далее, ${\displaystyle \forall y\in \left[c;d\right]\,\exists I\left(y\right)=\int \limits _{a}^{b}f\left(x,y\right)\,dx}$.
Функция ${\displaystyle I(y)}$ и называется интегралом, зависящим от параметра.

4. Свойства

\begin{itemize}
    \item \textbf{Непрерывность} 

    Пусть функция ${\displaystyle f(x,y)}$ непрерывна в области 
    ${\displaystyle {\overline {G}}}$ как функция двух переменных. 
    Тогда функция ${\displaystyle I\left(y\right)=\int \limits _{a}^{b}f\left(x,y\right)\,dx}$ 
    непрерывна на отрезке ${\displaystyle [c;d]}$.

    \item \textbf{Дифференцирование под знаком интеграла}
    Пусть теперь на области ${\displaystyle {\overline {G}}}$ 
    непрерывна не только функция ${\displaystyle f(x,y)}$, 
    но и её частная производная ${\displaystyle {\frac {\partial f}{\partial y}}\left(x,y\right)}$.
    Тогда ${\displaystyle {\frac {d}{dy}}I(y)=\int \limits _{a}^{b}{\frac {\partial f}{\partial y}}\left(x,y\right)\,dx}$

    \item \textbf{Интегрирование под знаком интеграла}

    Если функция ${\displaystyle f(x,y)}$ непрерывна в области ${\displaystyle {\overline {G}}}$, то
    ${\displaystyle \int \limits _{c}^{d}I(y)\,dy=\int \limits _{a}^{b}\left(\int \limits _{c}^{d}f(x,y)\,dy\right)\,dx}$
\end{itemize}
\section{
    Мера множества. Измеримые функции. Интеграл Лебега и его основные свойства.
}

1. \textbf{Мера множества}

Пусть задано множество ${\displaystyle X}$ с некоторым выделенным классом подмножеств 
${\displaystyle {\mathcal {F}}}$

Функция ${\displaystyle \mu \colon {\mathcal {F}}\to [0,\;\infty ]}$ называется мерой 
(иногда объёмом), если она удовлетворяет следующим аксиомам:
\begin{enumerate}
    \item ${\displaystyle \mu (\varnothing )=0}$ — мера пустого множества равна нулю;
    \item Для любых непересекающихся множеств ${\displaystyle A,B\in {\mathcal {F}},} {\displaystyle A\cap B=\varnothing }$
    
    $\mu (A\cup B)=\mu (A)+\mu (B)$ - мера объединения непересекающихся множеств равна сумме мер этих множеств (аддитивность, конечная аддитивность).
\end{enumerate}

2. \textbf{Измеримое множество}

Пусть ${\displaystyle (X,{\mathcal {F}})}$ и 
${\displaystyle (Y,{\mathcal {G}})}$ — два множества с выделенными алгебрами подмножеств. 

Тогда функция ${\displaystyle f:X\to Y}$ называется измеримой, если прообраз любого множества из ${\displaystyle {\mathcal {G}}}$ принадлежит 
${\displaystyle {\mathcal {F}}}$, то есть
${\displaystyle \forall B\in {\mathcal {G}},\;f^{-1}(B)\in {\mathcal {F}}}$

3. \textbf{Интеграл Лебега}

Дано пространство с мерой ${\displaystyle (X,{\mathcal {F}},\mu )}$, 
и на нём определена измеримая функция ${\displaystyle f\colon (X,{\mathcal {F}})\to (\mathbb {R} ,{\mathcal {B}}(\mathbb {R} ))}$.

Пусть ${\displaystyle f}$ — произвольная измеримая функция,  $A\in {\mathcal  {F}}$ 
произвольное измеримое множество. Тогда по определению
${\displaystyle \int \limits _{A}f(x)\,\mu (dx)=\int \limits _{X}f(x)\,\mathbf {1} _{A}(x)\,\mu (dx)}$, 
где ${\displaystyle \mathbf {1} _{A}(x)}$ — индикатор-функция множества ${\displaystyle A}$.


\section{
    Степенные ряды в действительной и комплексной области. \\
    Радиус сходимости. Теорема Коши-Адамара. Теорема Абеля. \\
    Свойства степенных рядов (почленное интегрирование и дифференцирование). \\
    Разложение элементарных функций. 
}

1. \textbf{Степенной ряд}

${\displaystyle F(X)=\sum \limits _{n=0}^{\infty }a_{n}X^{n}}$,в котором коэффициенты 
${\displaystyle {a_{n}}}$ берутся из некоторого кольца ${\displaystyle {R}}$.

2. \textbf{Радиус сходимости}

$\displaystyle R = \varlimsup _{n \to \infty} | \frac{a_n}{a_{n+1}}| = \varlimsup _{n \to \infty} \frac{1}{{\sqrt[{n}]{|a_n|}}}$ 
в круге которого $\displaystyle |x - x_0| < R$ ряд абсолютно сходится, а вне него расходится 

3. \textbf{Теорема Абеля}

Пусть ряд ${\displaystyle \Sigma \,a_{n}x^{n}}$ 
сходится в точке ${\displaystyle {x_{0}}}$. 
Тогда этот ряд сходится абсолютно в круге ${\displaystyle {|x|<|x_{0}|}}$
и равномерно по ${\displaystyle {x}}$ на любом компактном подмножестве этого круга.

4. \textbf{Теорема Коши-Адамара}

Пусть ${\displaystyle \sum _{\nu =0}^{+\infty }a_{\nu }(z-z_{0})^{\nu }}$
 — степенной ряд с радиусом сходимости ${\displaystyle R}$. Тогда:
 \begin{enumerate}
     \item если верхний предел ${\displaystyle \varlimsup \limits _{\nu \to +\infty }{\sqrt[{\nu }]{|a_{\nu }|}}}$
     существует и положителен, то ${\displaystyle R={\frac {1}{\varlimsup \limits _{\nu \to +\infty }{\sqrt[{\nu }]{|a_{\nu }|}}}}}$
     \item если ${\displaystyle \varlimsup \limits _{\nu \to +\infty }{\sqrt[{\nu }]{|a_{\nu }|}}=0}$, 
     то ${\displaystyle R=+\infty }$
     \item если верхнего предела ${\displaystyle \varlimsup \limits _{\nu \to +\infty }{\sqrt[{\nu }]{|a_{\nu }|}}}$
     не существует, то ${\displaystyle R=0}$.
 \end{enumerate}

5. Свойства
\begin{enumerate}
    \item \textbf{Дифференцирование}
    Пусть дан ряд $f$ с радиусом сходимости $R > 0$, функция $f$ непрерывна внутри круга:
    $ f'\left( x \right) = \large\frac{d}{{dx}}\normalsize{a_0} + \large\frac{d}{{dx}}\normalsize{a_1}x + \large\frac{d}{{dx}}\normalsize{a_2}{x^2} +  \ldots  = {a_1} + 2{a_2}x + 3{a_3}{x^2} +  \ldots  = \sum\limits_{n = 1}^\infty  {n{a_n}{x^{n - 1}}}$
    \item \textbf{Интегрирование}
    $\large\int\limits_b^x\normalsize {f\left( t \right)dt}  = \large\int\limits_b^x\normalsize {{a_0}dt}  + \large\int\limits_b^x\normalsize {{a_1}tdt}  + \large\int\limits_b^x\normalsize {{a_2}{t^2}dt}  +  \ldots  + \large\int\limits_b^x\normalsize {{a_n}{t^n}dt}  +  \ldots$
\end{enumerate}

\section{
    Функции комплексного переменного. Условия Коши-Римана. \\
    Геометрический смысл аргумента и модуля производной. 
}

1. Комплексные функции

Каждая комплексная функция ${\displaystyle w=f(z)=f(x+iy)}$
может рассматриваться как пара вещественных функций от двух переменных: 
${\displaystyle f(z)=u(x,y)+iv(x,y),}$ определяющих её вещественную и мнимую часть соответственно. 
Функции ${\displaystyle u,v}$ называются компонентами комплексной функции ${\displaystyle f(z)}$.

2. \textbf{Условия Коши-Римана}

Комплексная функция дифференцируема, если выполняется условия Коши-Римана:
$$
\frac{\partial u}{\partial x} = \frac{\partial v}{\partial y};
\frac{\partial u}{\partial y} = -\frac{\partial v}{\partial x}
$$

3. Аргумент и модуль производной
\begin{itemize}
    \item модуль определяет коэффициент маштабирования и определяет расстояние между точками в окрестности точки ${\displaystyle z}$
    \item аргумент определяет угол поворота гладкой кривой, проходящей через данную точку ${\displaystyle z}$
\end{itemize}


\section{
    Элементарные функции комплексного переменного $z^n,~e^z,~\frac{az+b}{ez+d},$ 
    и даваемые ими конформные отображения. 
    Простейшие многозначные $\sqrt{z},~\ln(z)$
}

?



\section{
    Теорема Коши об интеграле по замкнутому контуру. \\
    Интеграл Коши. Ряд Тейлора.
}

1. \textbf{Интегральная Теорема Коши}

Пусть ${\displaystyle D\subset \mathbb {C} }$ -- область, комплексная функция ${\displaystyle f(z)}$.
Тогда, если 
\begin{itemize}
    \item ${\displaystyle f(z)}$ голоморфна в ${\displaystyle D}$ и непрерывна в замыкании ${\displaystyle {\overline {D}}}$. 
    \item ${\displaystyle \Gamma}$ без петель
    \item ${\displaystyle \Gamma}$ конечна и без углов
\end{itemize}   
 справедливо соотношение 
${\displaystyle \oint \limits _{\Gamma }\,f(z)\,dz=0}$

2. \textbf{Интеграл Коши}

Пусть ${\displaystyle D}$ — область на комплексной плоскости с кусочно-гладкой границей 
${\displaystyle \Gamma =\partial D}$, функция ${\displaystyle f(z)}$ голоморфна в 
${\displaystyle {\overline {D}}}$, и ${\displaystyle z_{0}}$ — точка внутри области 
${\displaystyle D}$. Тогда справедлива следующая формула Коши:

${\displaystyle f(z_{0})={\frac {1}{2\pi i}}\int \limits _{\Gamma }{\frac {f(z)}{z-z_{0}}}\,dz}$

3. \textbf{Ряд Тейлора}

Рядом Тейлора в точке ${\displaystyle a}$ функции ${\displaystyle f(z)}$ комплексной переменной 
${\displaystyle z}$, удовлетворяющей в некоторой окрестности ${\displaystyle U\subseteq \mathbb {C}}$
точки ${\displaystyle a}$ условиям Коши — Римана, называется степенной Ряд

${\displaystyle f(z)=\sum _{n=0}^{+\infty }{\frac {f^{(n)}(a)}{n!}}(z-a)^{n}}.$
\section{
    Ряд Лорана. Полюс и существенно особая точка. \\
    Вычеты. Основная теорема о вычетах и ее применение. 
}

1. \textbf{Ряд Лорана} в конечной точке 
${\displaystyle z_{0}\in \mathbb {C} }$ — функциональный ряд по целым степеням ${\displaystyle (z-z_{0})}$
 над полем комплексных чисел:
${\displaystyle \sum _{n=-\infty }^{+\infty }c_{n}(z-z_{0})^{n},\quad }$
\begin{itemize}
    \item ряд ${\displaystyle \sum _{n=0}^{+\infty }c_{n}(z-z_{0})^{n}}$
    называется правильной частью,
    \item ряд ${\displaystyle \sum _{n=-\infty }^{-1}{c_{n}}{(z-z_{0})^{n}}}$
    называется главной частью.
\end{itemize}

2. Изолированная особая точка ${\displaystyle z_{0}}$ называется \textbf{полюсом} функции 
${\displaystyle f(z)}$, голоморфной в некоторой проколотой окрестности этой точки,
если существует предел ${\displaystyle \lim _{z\to {z_{0}}}f(z)=\infty }$

3. Для комплекснозначной функции ${\displaystyle f(z)}$ в области 
${\displaystyle D\subseteq \mathbb {C} }$, регулярной в некоторой проколотой
окрестности точки ${\displaystyle a\in D}$, её \textbf{вычетом} в точке ${\displaystyle a}$ называется число:

$${\displaystyle \mathop {\mathrm {Res} } _{a}\,f(z)=\lim _{\rho \to 0}{1 \over {2\pi i}}\int \limits _{|z-a|=\rho }\!f(z)\,dz}$$

4. \textbf{Основная теорема о вычетах}
Если функция ${\displaystyle f}$ аналитична в некоторой замкнутой односвязной области 
${\displaystyle {\overline {G}}\subset \mathbb {C} }$, за исключением конечного числа 
особых точек ${\displaystyle a_{1},a_{2},\dots ,a_{n}}$, из которых ни одна не принадлежит
граничному контуру ${\displaystyle \partial G}$, то справедлива следующая формула:

$${\displaystyle ~\int \limits _{\partial G}f(z)\,dz=2\pi i\sum _{k=1}^{n}\mathop {\mathrm {res} } _{z=a_{k}}f(z),}$$
где ${\displaystyle \mathop {\mathrm {res} } _{z=a_{k}}f}$ — вычет функции ${\displaystyle f}$ в точке ${\displaystyle a_{k}}$.
\section{
    Линейные преобразования. Квадратичные формы. \\
    Приведение их к каноническому виду линейными преобразованиями в комплексной и \\
    действительной областях. Закон инерции.
}

1. \textbf{Линейное отображение}

Линейным отображением векторного пространства ${\displaystyle V}$ 
над полем ${\displaystyle K}$ в векторное пространство 
${\displaystyle W}$ над тем же полем ${\displaystyle K}$ 
(линейным оператором из ${\displaystyle V}$ в ${\displaystyle W}$) 
называется \textbf{отображение} ${\displaystyle f\colon V\to W}$, удовлетворяющее условию линейности:

\begin{itemize}
    \item ${\displaystyle f(x+y)=f(x)+f(y)}$,
    \item ${\displaystyle f(\alpha x)=\alpha f(x)}$.
    
    для всех ${\displaystyle x,y\in V}$ и ${\displaystyle \alpha \in K}$.
\end{itemize}

Если ${\displaystyle V}$ и ${\displaystyle W}$ — это одно и то же векторное пространство, 
то ${\displaystyle f}$ — не просто линейное отображение, а \textbf{линейное преобразование}.

2. Пусть ${\displaystyle L}$ есть векторное пространство над полем 
${\displaystyle K}$ и ${\displaystyle e_{1},e_{2},\dots ,e_{n}}$ — базис в ${\displaystyle L}$.
Функция ${\displaystyle Q:L\to K}$ называется \textbf{квадратичной формой}, если её можно представить в виде:

${\displaystyle Q(x)=\sum _{i,j=1}^{n}a_{ij}x_{i}x_{j},}$ где 
${\displaystyle x=x_{1}e_{1}+x_{2}e_{2}+\cdots +x_{n}e_{n}}$, а 
${\displaystyle a_{ij}}$ — некоторые элементы поля ${\displaystyle K}$.

3. Метод Лагранжа приведения к каноническому виду

Каннонический вид: $\displaystyle f(X) = \lambda_1x_1^2 + \lambda_2x_2^2 + ... + \lambda_nx_n^2$

Пошаговое преобразование к канноническому виду методом Лагранжа:

Для $x_1$:

$f(x_{1},x_{2},...,x_{n})=(a_{{11}}x_{1}^{2}+2a_{{12}}x_{1}x_{2}+...+2a_{{1n}}x_{1}x_{n})+f_{1}(x_{2},x_{3},...,x_{n})= \\
{\displaystyle ={\frac {1}{a_{11}}}(a_{11}x_{1}+a_{12}x_{2}+...+a_{1n}x_{n})^{2}-{\frac {1}{a_{11}}}(a_{12}x_{2}+...+a_{1n}x_{n})^{2}+f_{1}(x_{2},x_{3},...,x_{n})=} \\
={\frac  {1}{a_{{11}}}}(a_{{11}}x_{1}+a_{{12}}x_{2}+...+a_{{1n}}x_{n})^{2}-{\frac  {1}{a_{{11}}}}(a_{{12}}x_{2}+...+a_{{1n}}x_{n})^{2}+f_{1}(x_{2},x_{3},...,x_{n})= \\
{\displaystyle ={\frac {1}{a_{11}}}y_{1}^{2}+f_{2}(x_{2},x_{3},...,x_{n})}={\frac  {1}{a_{{11}}}}y_{1}^{2}+f_{2}(x_{2},x_{3},...,x_{n})$

4. \textbf{Закон инерции}

Число положительных, отрицательных и нулевых канонических коэфициентов квадратичной формы 
не зависит от преобразования, с помощью которого квадатичная форма приводится к каноническому виду
\section{
    Линейная зависимость и независимость векторов. Ранг матрицы. \\
    Системы линейных алгебраических уравнений, теорема Кронекера-Капелли. \\ 
    Общее решение системы линейных алгебраических уравнений. 
}

1. Линейная зависимость и независимость векторов

Конечное множество ${\displaystyle M'=\{v_{1},v_{2},...,v_{n}\}}$ называется линейно
 независимым, если единственная линейная комбинация, равная нулю, тривиальна, 
 то есть состоит из факторов, равных нулю:
${\displaystyle a_{1}v_{1}+a_{2}v_{2}+\ldots +a_{n}v_{n}=0\quad \Rightarrow \quad a_{1}=a_{2}=\ldots =a_{n}=0.}$

Если существует такая линейная комбинация с минимум одним
${\displaystyle a_{i}\neq 0}a_{i}\neq 0, {\displaystyle M'}$ называется линейно зависимым.

2. \textbf{Ранг}

Пусть ${\displaystyle A_{m\times n}}$ — прямоугольная матрица. Рангом матрицы ${\displaystyle A}$ является:
\begin{itemize}
    \item ноль, если ${\displaystyle A}$ — нулевая матрица;
    \item число ${\displaystyle r\in \mathbb {N} :\;\exists M_{r}\neq 0,\;\forall M_{r+1}=0}$, 
    где ${\displaystyle M_{r}}$ — минор матрицы ${\displaystyle A}$ порядка ${\displaystyle r}$, 
    а ${\displaystyle M_{r+1}}$ — окаймляющий к нему минор порядка ${\displaystyle (r+1)}$, 
    если они существуют.
\end{itemize}

3. Теорема Кронекера-Капелли

Система линейных алгебраических уравнений совместна тогда и только тогда, когда ранг её основной матрицы равен рангу её расширенной матрицы.
\section{
    Ортогональные преобразования в евклидовом пространстве \\
    и ортогональные матрицы. Свойства ортогональных матриц.
}

1. \textbf{Ортогональное преобразование} — линейное преобразование 
${\displaystyle A}$ евклидова пространства ${\displaystyle L}$, 
сохраняющее длины или (что эквивалентно) скалярное произведение векторов. 
Это означает, что для любых двух векторов ${\displaystyle x,y\in L}$ выполняется равенство

${\displaystyle \langle A(x),\,A(y)\rangle =\langle x,\,y\rangle}$

2. \textbf{Ортогональная матрица} $A$ такая, что $AA^{{T}}=A^{{T}}A=E$

3. Свойства
\begin{itemize}
    \item Ортогональная матрица является унитарной 
    ${\displaystyle Q^{-1}=Q^{*}}$ и, следовательно, нормальной 
    ${\displaystyle Q^{*}Q=QQ^{*}}$
    \item Cкалярное произведение строки на саму себя равно 1, а на любую другую строку — 0. Это же справедливо и для столбцов.
    
    \item Определитель ортогональной матрицы равен ${\displaystyle \pm 1}$, 
    \item Множество ортогональных матриц порядка ${\displaystyle n}$ 
    над полем ${\displaystyle k}$ образует группу по умножению, 
    так называемую ортогональную группу которая обозначается ${\displaystyle O_{n}(k)}$ 
    или ${\displaystyle O(n,\;k)}$ (если ${\displaystyle k}$ опускается, 
    то предполагается ${\displaystyle k=\mathbb {R} }$.
    \item Линейный оператор, заданный ортогональной матрицей, 
    переводит ортонормированный базис линейного пространства в ортонормированный.
    \item Матрица вращения является специальной ортогональной (квадратная, оперделитель 1). 
    Матрица отражения является ортогональной.
    \item Любая вещественная ортогональная матрица подобна 
    (одного порядка и существует P того же порядка, такая что: ${\displaystyle B=P^{-1}AP}$)
    блочно-диагональной матрице с блоками вида $(\pm 1)$ и 
    ${\displaystyle {\begin{pmatrix}\ \ \ \cos \varphi &\sin \varphi \\-\sin \varphi &\cos \varphi \end{pmatrix}}.}$
\end{itemize}
\section{
    Характеристический многочлен линейного преобразования векторного \\
    пространства. Собственные числа и собственные векторы. \\
    Свойства собственных чисел и векторов симметрических матриц. \\
    Понятие о методе ортогональных вращений решения полной проблемы собственных значений. 
}

\part{Дополнительная часть}

\section{Элементы теории алгоритмов, математической логики и дискретной математики}

\subsection{Понятие алгоритма и его уточнения. Машины Тьюринга, нормальные алгоритмы Маркова; эквивалентность этих уточнений. Тезис Тьюринга. Понятие алгоритмической неразрешимости; примеры алгоритмически неразрешимых проблем.}
\subsection{Временная и пространственная алгебраическая сложность алгоритма в худшем случае и в среднем; символы $\mathbf{O}$, $\Omega$, и $\Theta$ в асимптотических оценках сложности. Сложность в худшем случае сортировки слияниями, быстрой сортировки; сложность в среднем быстрой сортировки. Понятие битовой сложности. Нижние границы сложности класса алгоритмов. Нижние границы сложности классов алгоритмов сортировки и поиска с помощью сравнений. Алгоритм, оптимальный в данном классе, оптимальность по порядку сложности.}
\subsection{Классы Р и NP алгоритмов распознавания языков. Проблема P $\stackrel{?}{=}$ NP . Полиномиальная сводимость; NP -полные задачи (формулировка основных фактов, примеры).}
\subsection{Логика 1-го порядка. Выполнимость и общезначимость. Общая схема метода резолюций. Логические программы. SLD-резолютивные вычисления логических программ. Правильные и вычислимые ответы на запросы к логическим программам. Стандартная стратегия выполнения логических программ.}
\subsection{Теорема Поста о полноте систем функций в алгебре логики. Графы, деревья, планарные графы, их свойства. }
\end{document}
